\section{Discussion}

It  is  worth  noting  that the calculations in this report assume  the  worst
possible  case  (the  voltage is a step function)  and  this  may  not  be  an
assumption  you  can  make  in real world applications. A stepped voltage  can
appear for example in a sample and hold (S\&H) circuit of an ADC converter, so
in that situation -- if efficiency is a concern -- the  calculations  in  this
report are relevant.

The  transfer  efficiency  will  \textbf{increase}  substantially  for  slower
changes  in  the  input  voltage  ($\frac{dv}{dt}$).  Bypass  capacitors,  for
example, are powered by voltage  regulators  that  typically  have fairly long
ramp-up  times  (in  the  order  of  multiple  milli-seconds), so the transfer
efficiency  in  these cases will be a lot better and circuits such as the  one
seen  in  figure \ref{fig:bypass} won't dissipate any  significant  amount  of
energy.

It is also worth pointing out that adding a series inductor to the circuit, as
seen in figure \ref{fig:series-inductor} has no  influence on the total amount
of dissipated energy, even if the circuit oscillates.

Letting     $R\to    0$    in    equations    \ref{eq:resistor-energy2}    and
\ref{eq:capacitor-energy2} does not converge to  any  limit.  It  is  still an
interesting thought experiment to think about what might happen if one had two
superconducting capacitors and were able to close  a switch close to the speed
of  light. An educated guess might be that some of the  energy  is  dissipated
over  the  electric arc formed between the switch contacts as it closes,  and,
assuming  the  switch  is  able  to  close  before all of the  energy  can  be
transferred (limited by the transmission speed of  electrons),  the  remaining
energy  would  then  oscillate  back  and forth between  the  two  capacitors,
assuming the  superconducting wire connecting them is slightly inductive. It's
possible  that  the  energy  would  slowly decay over time as  electromagnetic
radiation.

A  far  more  complete  explanation  of what happens for $R\to 0$ as  well  as
analogies  to  water  tanks and thought experiments where the capacitor plates
are    stretched     to    decrease    their    voltages    can    be    found
here\cite{ref:arXiv:1309.5034}.

\newpage

\begin{figure}[t]
\centering
\begin{circuitikz} \draw
    (0,0) node[rground]{}
    (0,0) to[C=$10\micro F$,-*]  (0,2)
          to[generic=$Z$]        (3,2)
          to[C=$10\micro F$]     (3,0)
          node[rground]{}
    (0,2) node[vcc]{$V_{cc}$}
    (3,2) to[short,i=to load,*-] (4.5,2)
    ;
\end{circuitikz}
    \caption{A common way to bypass a load: Two capacitors and a ferrite bead (sometimes, inductors or even resistors are used in place of the ferrite bead).}
    \label{fig:bypass}
\end{figure}

\begin{figure}[t]
\centering
\begin{circuitikz} \draw
    (0,0) to[C=$C_1$]     (0,2.5)
          to[spst=$S_1$]  (1.5,2.5)
          to[L=$L$]       (3,2.5)
          to[R=$R$]       (5,2.5)
          to[C=$C_2$]     (5,0)
          to[short]       (0,0)
    ;
\end{circuitikz}
    \caption{Inductor $L$ has no influence on any of the previous calculations.}
    \label{fig:series-inductor}
\end{figure}

\vspace*{2cm}

