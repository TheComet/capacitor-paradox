\subsection{The Easy Way}

Perhaps the easiest explanation  of  all  is  to realise that the model can be
further simplified into an equivalent circuit. Consider the  circuits shown in
figures \ref{fig:rearrange} and \ref{fig:equivalent-circuit}.

\begin{figure}[th!]
\centering
\begin{circuitikz} \draw
    (0,0) to[C=$C_2$]          (0,2)
          to[C=$C_1$]          (0,4)
          to[spst=$S_1$]       (3,4)
          to[R=$R$,i>^=$i(t)$] (3,0)
          to                   (0,0)
    (0.5,4) to[open,v^=$\Delta V$] (0.5,0)
    ;
\end{circuitikz}
    \caption{Re-arrangement of the same circuit seen in figure \ref{fig:circuit}.}
    \label{fig:rearrange}
\end{figure}

\begin{figure}[th!]
\centering
\begin{circuitikz} \draw
    (0,0) to[C=$C_{eq}$,v<=$\Delta V$] (0,3)
          to                           (3,3)
          to[R=$R$,i>^=$i(t)$]         (3,0)
          to                           (0,0)
    ;
\end{circuitikz}
    \caption{Equivalent circuit of figure \ref{fig:rearrange}}
    \label{fig:equivalent-circuit}
\end{figure}

The equivalent circuit seen in figure  \ref{fig:equivalent-circuit}  makes  it
very easy to  see  that  all of the energy in $C_{eq}$ will discharge over $R$
and dissipate. The calculation is:

\begin{align}
    e_{loss} &= \frac{1}{2} C_{eq} {\Delta V}^2 \\
             &= \frac{1}{2} \frac{C_1C_2}{C_1+C_2} \left(V_{0,C_1} - V_{0,C_2}\right)^2
\end{align}

Using the  same  numbers  from  the  first  example,  we  find  that $C_{eq} =
\SI{500}{\nano\farad}$,  $\Delta  V   =   \SI{10}{\volt}$   and   $e_{loss}  =
\SI{25}{\micro\joule}$, as expected.

