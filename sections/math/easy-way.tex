\subsection{The Easy Way}

Perhaps the easiest explanation  of  all  is  to realise that the model can be
further simplified into an equivalent circuit. Consider the  circuits shown in
figures \ref{fig:rearrange} and \ref{fig:equivalent-circuit}.

The equivalent circuit seen in figure  \ref{fig:equivalent-circuit}  makes  it
very easy to  see  that  all of the energy in $C_{eq}$ will discharge over $R$
and dissipate. The calculation is:

\begin{align}
    e_{loss} &= \frac{1}{2} C_{eq} {\Delta V}^2 \\
             &= \frac{1}{2} \frac{C_1C_2}{C_1+C_2} \left(V_{0,C_1} - V_{0,C_2}\right)^2
\end{align}

Using the  same  numbers  from  the  first  example,  we  find  that $C_{eq} =
\SI{500}{\nano\farad}$,  $\Delta  V   =   \SI{10}{\volt}$   and   $e_{loss}  =
\SI{25}{\micro\joule}$, as expected.

\newpage

\begin{figure}[t]
    \centering
    \begin{circuitikz} \draw
        (0,0) to[C=$C_1$,v<=$V_{0,C_1}$,i=$i(t)$] (0,3)
              to[spst]                            (1.5,3)
              to[R=$R$,v=$V_R$,i=$i(t)$]          (4,3)
              to[C=$C_2$,v=$V_{0,C_2}$,i>=$i(t)$] (4,0)
              to                                  (0,0)
        ;
    \end{circuitikz}
    \caption{}
    \label{fig:circuit}
\end{figure}

\begin{figure}[t]
    \centering
    \begin{subfigure}[b]{\linewidth}
        \centering
        \begin{circuitikz} \draw
            (0,0) to[C=$C_2$]          (0,2)
                  to[C=$C_1$]          (0,4)
                  to[spst=$S_1$]       (3,4)
                  to[R=$R$,i>^=$i(t)$] (3,0)
                  to                   (0,0)
            (0.5,4) to[open,v^=$\Delta V$] (0.5,0)
            ;
        \end{circuitikz}
        \caption{}
        \label{fig:rearrange}
    \end{subfigure}
    \begin{subfigure}[b]{\linewidth}
        \centering
        \begin{circuitikz} \draw
            (0,0) to[C=$C_{eq}$,v<=$\Delta V$] (0,3)
                  to                           (3,3)
                  to[R=$R$,i>^=$i(t)$]         (3,0)
                  to                           (0,0)
            ;
        \end{circuitikz}
        \caption{}
        \label{fig:equivalent-circuit}
    \end{subfigure}
    \caption{Shows \textbf{(a)} a re-arrangement of the circuit from figure \ref{fig:circuit} and \textbf{(b)} a simplified version of the circuit.}
\end{figure}

\vspace*{2cm}
