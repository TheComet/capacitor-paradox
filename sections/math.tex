\section{Math}

\subsection{Deriving Current}

\begin{figure}[th!]
\begin{circuitikz} \draw 
    (0,0) to[C=$C_1$,v<=$V_{0,C_1}$,i=$i(t)$] (0,3)
          to[spst]                            (1.5,3)
          to[R=$R$,v=$V_R$,i=$i(t)$]          (4,3)
          to[C=$C_2$,v=$V_{0,C_2}$,i>=$i(t)$] (4,0)
          to                                  (0,0)
    ;
\end{circuitikz}
    \caption{}
    \label{fig:circuit}
\end{figure}

Closing the switch  at  $t=0$ completes the circuit and Kirchhoff's second law
must hold true:

\begin{align}
    \sum_{k=1}^{N} V_k &= 0 \\
    V_{C_1}(t) &= V_R(t) + V_{C_2}(t)
    \label{eq:khv}
\end{align}

Kirchoff's first  law  also  states  that  the  current  through  the resistor
$i_R(t)$ must be equal to the current leaving and entering the  two capacitors
$i_{C_1}(t)$ and $i_{C_2}(t)$.

\begin{equation}
    i(t) = i_R(t) = i_{C_1}(t) = i_{C_2}(t)
    \label{eq:khi}
\end{equation}

In  order to solve for the unknown function  $i(t)$,  the  voltage  over  each
element can be expressed as a function of current:

\begin{align}
    V_R    (t) &= i(t)R \label{eq:vr}\\
    V_{C_1}(t) &= -\frac{1}{C_1}\int_0^t i(\tau)\,d\tau + V_{0,C_1} \label{eq:vc1}\\
    V_{C_2}(t) &= \frac{1}{C_2}\int_0^t i(\tau)\,d\tau + V_{0,C_2} \label{eq:vc2}
\end{align}

Note  the negation in equation \ref{eq:vc1}, because the voltage $V_{C_1}$  is
drawn in the reverse direction in figure \ref{fig:circuit}.

By substituting the equations  \ref{eq:vr}, \ref{eq:vc1} and \ref{eq:vc2} into
equation \ref{eq:khv} we obtain the differential equation

\begin{equation}
    \frac{1}{C_1}\int_0^t i(\tau)\,d\tau + \Delta V = i(t)R + \frac{1}{C_2}\int_0^t i(\tau)\,d\tau
    \label{eq:diff}
\end{equation}

where $\Delta V = V_{0,C_1} - V_{0,C_2}$.

The differential  equation  can  be solved by using the Laplace transform with
the following relationships.

\begin{align}
    \laplace{af(t)} &= aF(s), & f(t<0)=0 \\
    \laplace{a} &= \frac{a}{s}, & a > 0 \\
    \laplace{e^{-at}} &= \frac{1}{s+a}, & a > 0 \label{eq:expdecay} \\
    \laplace{\int_0^t f(\tau)\,d\tau} &= \frac{F(s)}{s}, & f(\tau<0)=0
\end{align}

Transforming equation \ref{eq:diff} results yields:

\begin{align}
    \frac{I(s)}{sC_1} &= I(s)R + \frac{I(s)}{sC_2} - \frac{\Delta V}{s}\\
    \frac{\Delta V}{s} &= I(s)\frac{sRC_2 + 1}{sC_2} + \frac{I(s)}{sC_1} \\
    \Delta V &= I(s)\frac{sRC_1C_2 + C_1 + C_2}{C_1C_2} \\
    I(s) &= \frac{\Delta V C_1 C_2}{sRC_1C_2 + C_1 + C_2} \\
    I(s) &= \frac{\Delta V}{R}\cdot\frac{1}{s + \frac{C_1 + C_2}{RC_1C_2}} \label{eq:laplaceresult}
\end{align}

Equation \ref{eq:laplaceresult} can  be  transformed back into the time domain
by using the relationship described by  equation  \ref{eq:expdecay}, and so we
obtain the well-known current formula for an RC-Network.

\begin{equation}
    i(t) = I_0 \cdot e^{\frac{-t}{\tau}} \hspace{5mm} \begin{aligned}
        I_0  &= \frac{V_{0,C_1}-V_{0,C_2}}{R} \\
        \tau &= R \frac{C_1C_2}{C_1+C_2}
    \end{aligned}
    \label{eq:currentresult}
\end{equation}

This result is intuitive, because from the perspective of the resistor (with a
closed   switch),   we   effectively   ``see''   a   total    capacitance   of
$\frac{C_1C_2}{C_1+C_2}$  charged  to  a voltage of $V_{0,C_1}  -  V_{0,C_2}$.

\subsection{Energy Dissipated by the Resistor}

The power and  energy  dissipated  by  the resistor can be calculated from the
current $i(t)$:

\begin{align}
    p_R(t) &= i^2(t) R \label{eq:power} \\
    e_R(t) &= \int_0^t p_R(\tau)\,d\tau \label{eq:energy}
\end{align}

Substituting  the equation for $i(t)$ \ref{eq:currentresult}  into  the  power
equation \ref{eq:power} yields:

\begin{align}
    p_R(t) &= R\left(\frac{\Delta V}{R} e^{\frac{-t}{\tau}}\right)^2 \\
           &= \frac{{\Delta V}^2}{R} e^{\frac{-2t}{\tau}}
\end{align}

Substituting  this  equation  into the  energy  equation  \ref{eq:energy}  and
solving the integral yields:

\begin{align}
    e_R(t) &= \frac{{\Delta V}^2}{R} \int_0^t e^{\frac{-2z}{\tau}}\,dz \\
           &= \frac{{\Delta V}^2}{R} \left(-\frac{\tau}{2} e^{\frac{-2z}{\tau}} \bigg|_0^t\right) \\
           &= \frac{{\Delta V}^2}{R} \frac{\tau}{2}\left(1 - e^{\frac{-2t}{\tau}}\right)
\end{align}

The resulting expression can further be simplified by substituting $\tau$ with
the result from \ref{feq:currentresult}:

\begin{align}
    e_R(t) &= \frac{{\Delta V}^2}{R} \frac{RC_1C_2}{2\left(C_1+C_2\right)}\left(1 - e^{\frac{-2t}{\tau}}\right) \\
           &= \frac{1}{2} \frac{C_1C_2}{C_1+C_2} {\Delta V}^2 \left(1 - e^{\frac{-2t\left(C_1+C_2\right)}{RC_1C_2}}\right)
\end{align}

If we allow the circuit to settle for a significant amount of time, it is easy
to see that  the  energy dissipated by the resistor \textbf{does not depend on
the value of the resistor!}

\begin{align}
    \lim_{R\to\infty} & \frac{1}{2} \frac{C_1C_2}{C_1+C_2} {\Delta V}^2 \left(1 - e^{\frac{-2t\left(C_1+C_2\right)}{RC_1C_2}}\right) \\
    &= \frac{1}{2} \frac{C_1C_2}{C_1+C_2} {\Delta V}^2 \label{eq:finalresult}
\end{align}

\subsection{Energy Transfer of the Capacitors}

The  current function  $i(t)$  from  equation  \ref{eq:currentresult}  can  be
re-inserted  into  the integral equations  \ref{eq:vc1}  and  \ref{eq:vc2}  to
calculate the voltages on both capacitors.

\begin{align}
    v(t) &= \frac{I_0}{C} \int_0^t e^\frac{-z}{\tau}\,dz + V_0\\
         &= -\frac{I_0\tau}{C} \cdot e^\frac{-z}{\tau} \bigg|_0^t + V_0 \\
         &= \frac{I_0\tau}{C}\left(1-e^\frac{-t}{\tau}\right) + V_0
\end{align}

The power is of course

\begin{align}
    p(t) &= u(t)\cdot i(t) \\
         &= \left( \frac{I_0\tau}{C}\left(1 - e^\frac{-t}{\tau}\right) + V_0 \right) \cdot I_0\,e^\frac{-t}{\tau} \\
         &= \frac{I_0^2\tau}{C}\left(e^\frac{-t}{\tau}-e^\frac{-2t}{\tau}\right) + V_0I_0\,e^\frac{-t}{\tau}
\end{align}

and the energy is therefore

\begin{align}
    e(t) &= \int_0^z p(t)\,dz \\
         & \begin{aligned}
            = \frac{I_0^2\tau}{C}\int_0^t e^\frac{-z}{\tau}\,dz &- \frac{I_0^2\tau}{C}\int_0^t e^\frac{-2z}{\tau}\,dz \\
                                                               &+ U_0I_0\int_0^t e^\frac{-z}{\tau}\,dz
        \end{aligned} \\
         &= 
\end{align}

\subsection{eh}

In fact, the resulting equation \ref{eq:finalresult} is identical to the formula used to calculate the \textit{initial energy difference of the capacitors!} For $t<0$, the switch is open and both capacitors are connected in series. The total capacitance is therefore

\begin{equation}
    C_{tot} = \frac{C_1C_2}{C_1+C_2}
\end{equation}

The capacitors are charged with an initial voltage of $V_{0,C_1}$ and $V_{0,C_2}$. Because $V_{0,C_1}$ is drawn in figure \ref{fig:circuit} with opposite polarity to $V_{0,C_2}$, the total initial voltage over both capacitors is $\Delta V = V_{0,C_1} - V_{0,C_2}$ and the total initial energy difference is therefore:

\begin{equation}
    e_{C_{tot}} = \frac{1}{2} C_{tot} {\Delta V}^2
\end{equation}

The surprising result is that $e_{C_{tot}} = e_R(t)\bigg|_{t\to\infty}$ or in other words: The initial energy stored in both capacitors is equal to the energy dissipated by the resistor after closing the switch and letting the circuit settle.
