\section{Introduction}

The \textit{Capacitor  Paradox}  is  a  widely  known  trick question commonly
encountered   in   the   field   of   electronics,   especially  on   internet
forums\cite{ref:physicsforums},  and sometimes even in electrical  engineering
lectures, designed to throw people off.

Consider   two   capacitors,   $C_1   =  \SI{1}{\micro\farad}$  and   $C_2   =
\SI{1}{\micro\farad}$, with initial voltages of $V_{0,C_1}=\SI{10}{\volt}$ and
$V_{0,C_2}=\SI{0}{\volt}$,  respectively  (see   figure   \ref{fig:two-caps}).

The energy stored in $C_1$ is

\begin{equation}
    e_{C_1} = \frac{1}{2} C_1 V_{0,C_1}^2 = \SI{50}{\micro\joule}
\end{equation}

Now,  close  the switch $S_1$ and let the circuit settle. Since $C_1=C_2$, the
charge from $C_1$ will evenly distribute over both capacitors  such  that both
voltages settle at $V_{settle} = \SI{5}{\volt}$.

Calculating the new energy stored in both capacitors reveals something strange:

\begin{align}
    e_{settle} &= 2\left(\frac{1}{2} C_1 V_{settle}\right) \\
               &= 2\cdot\SI{12.5}{\micro\joule} \\
               &= \SI{25}{\micro\joule} \\
\end{align}

The  new  total energy $e_{settle}$ is \textbf{half} of what  it  was  before.
Where did the energy go?

\begin{figure}[h!]
\centering
\begin{circuitikz} \draw
    (0,0) to[C=$C_1$,v<=$V_{0,C_1}$] (0,3)
          to[spst=$S_1$]             (4,3)
          to[C=$C_2$,v=$V_{0,C_2}$]  (4,0)
          to                         (0,0)
    ;
\end{circuitikz}
    \caption{Closing the switch discharges $C_1$ into $C_2$. Calculating the new total amount of energy reveals that half of the energy has gone missing. Why?}
    \label{fig:two-caps}
\end{figure}

